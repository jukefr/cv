% !TEX program = xelatex

\documentclass{resume}

\begin{document}
\pagenumbering{gobble} % suppress displaying page number

\name{Kevin Jullien}

\basicInfo{\email{hi@juke.fr} \textperiodcentered\ 
  \phone{(+33) 7 51 07 30 64} \textperiodcentered\ 
  \linkedin[kjullien]{https://www.linkedin.com/in/kjullien}}

\basicInfo{
  \role{}{Développeur Node.js}}

\section{\faGraduationCap\ Education}
\datedsubsection{\textbf{Web School Factory}, Paris, France}{2015 -- 2017-05}
Master en management digital (spécialité Technologies Numériques)
\datedsubsection{\textbf{Ombrosa Multilingual School}, Lyon, France}{2012 -- 2015}
Baccalauréat scientifique

\section{\faBuilding\ Experience}
\datedsubsection{\textbf{Travelcar} Paris, France}{2017-06 -- 2019-01}
\role{CDI}{Concepteur Développeur}

Chargé de la conception, développement, mise en place et maintenance d'un CMS pour environ 1 000 marques et sous-marques.
Occasionnellement assigné à des taches d'expérimentation pour optimiser les processus internes.
\begin{itemize}
  \item \textbf{Conception d'infrastructure}, Wordpress hébergé sur EC2, scale avec Elastic Beanstalk. \textit{[PHP/Docker/AWS]}
  \item \textbf{Conception 4 thèmes} pour marques et sous marques avec branding configurable en back-office. \textit{[PHP]}
  \item \textbf{Conception de services} pour synchronisation APIs internes vers Wordpress. \textit{[PHP]}
  \item \textbf{Maintenance et mise en place nouvelles fonctionnalités sur les Wordpress}. \textit{[PHP]}
  \item \textbf{Expériences avec machine learning}, réduire outsourcing de vérifications d'images, nouvelles formes \newline d'authentification, etc. \textit{[Python]}
  \item \textbf{Divers scripts et CLIs} pour améliorer l'experience de développement. \textit{[Typescript]}
\end{itemize}


\section{\faGithub\ Loisirs}
\datedsubsection{\textbf{Open Source}}{2012 -- 2019}
\role{temps libre}{GitHub}
\begin{itemize}
  \item \textbf{fyle} - utilitaires conteneurisés d'optimisation, compression et plus pour médias web. \textit{[Docker/Shell]}
  \item \textbf{Lardons} - infrastructure conteneurisée fullstack avec backend Strapi (WordPress des APIs), frontend Vue.js. \textit{[docker-compose/Vue.js/Mongo/Node]}
  \item \textbf{clize} - utilitaire Node.js pour tester des methodes Javascript directement depuis sa console. \textit{[Typescript]}
  \item \textbf{nodend.com} - blog de Technologies sur divers sujets, Linux, Node, Bash etc. \textit{[Go]}
  \item \textbf{actions} - actions conteneurisées pour Github Actions. \textit{[Docker/Bash]}
  \item \textbf{ngrok-service} - utilitaire d'installation interactif Ngrok2 via System.d. \textit{[Bash]}
  \item \textbf{Micro-Wordpress} - framework de micro-services, un fichier php = un service, exposé automatiquement via l'API REST Wordpress. \textit{[PHP]}
  \item \textbf{CV} - ce cv est aussi disponible sur mon GitHub. \textit{[LaTeX]}
\end{itemize}

\section{\faCogs\ Talents}
\begin{itemize}[parsep=0.5ex]
  \item \textbf{Languages}: Node \textbf{=} Typescript \textbf{=} Javascript \textbf{=} ES6 \textbf{>} PHP \textbf{=} Python \textbf{>} Go
  \item \textbf{Platformes}: Linux \textbf{=} macOs \textbf{>} Windows
  \item \textbf{Développement}: Serveur \textbf{=} OPS \textbf{>} Client
\end{itemize}

\section{\faInfo\ Autre}
\begin{itemize}[parsep=0.5ex]
  \item \textbf{Blog}: https://nodend.com
  \item \textbf{GitHub}: https://github.com/jukefr
  \item \textbf{Languages}: English (C2) = Français (C2) > Español (A2)
\end{itemize}

\end{document}
