% !TEX program = xelatex

\documentclass{resume}
%\usepackage{zh_CN-Adobefonts_external} % Simplified Chinese Support using external fonts (./fonts/zh_CN-Adobe/)
%\usepackage{zh_CN-Adobefonts_internal} % Simplified Chinese Support using system fonts

\begin{document}
\pagenumbering{gobble} % suppress displaying page number

\name{Kevin Jullien}

\basicInfo{\email{hi@juke.fr} \textperiodcentered\ 
  \phone{(+33) 7 51 07 30 64} \textperiodcentered\ 
  \linkedin[kjullien]{https://www.linkedin.com/in/kjullien}}

\section{\faGraduationCap\ Education}
\datedsubsection{\textbf{Web School Factory}, Paris, France}{2015 -- 2017}
Master en management digital (spécialité Technologies Numériques), abandonné pour poursuivre carrière professionnelle.
\datedsubsection{\textbf{Ombrosa Multilingual School}, Lyon, France}{2012 -- 2015}
Baccalauréat scientifique

\section{\faUsers\ Experience}
\datedsubsection{\textbf{Travelcar} Paris, France}{2017 -- 2019}
\role{Concepteur Développeur}

Chargé de la conception, développement, mise en place et maintenance d'un CMS pour environ 1 000 marques et sous-marques.
Occasionnellement assigné à des taches d'expérimentation pour optimiser les processus internes. (Machine learning, systèmes homemade, etc.)
\begin{itemize}
  \item \textbf{Conception d'infrastructure}, Wordpress hébergé sur EC2, scale avec Elastic Beanstalk. \textit{[PHP/Docker/AWS]}
  \item \textbf{Conception 4 thèmes} pour marques et sous marques Travelcar avec branding configurable en back-office. (Marques grises etc.) \textit{[PHP]}
  \item Mise en place d'une \textbf{synchronisation des APIs Travelcar} pour créer et mettre à jour les pages des Wordpress automatiquement. \textit{[PHP/Node/Python]}
  \item \textbf{Maintenance et mise en place nouvelles fonctionnalités sur les Wordpress}. \textit{[PHP]}
  \item \textbf{Expériences avec machine learning} (réduire outsourcing de vérifications d'images, nouvelles formes \newline d'authentification, etc.). \textit{[Python]}
  \item \textbf{Divers scripts et CLIs} pour faciliter l'UX de développement. \textit{[Typescript]}
\end{itemize}

\datedsubsection{\textbf{Open Source}}{2012 -- 2019}
\role{Github}{}
J'affiche fièrement mes contributions open source et ne jure que par le "free/libre software".
\begin{itemize}
  \item \textbf{fyle} - utilitaires de processing d'assets web (optimisation, compression, etc.) \textit{[Docker/Shell]}
  \item \textbf{Lardons} - boilerplate fullstack Strapi, Mongo, Nuxt et Docker \textit{[docker-compose/Vue.js/Mongo/Node]}
  \item \textbf{clize} - utilitaire node pour executer des exports directement depuis la console \textit{[Typescript]}
  \item \textbf{nodend.com} - blog personnel d'articles/tutos divers sur Linux, Node, Bash etc. \textit{[Go]}
  \item \textbf{actions} - actions pour Github Actions \textit{[Docker/Bash]}
  \item \textbf{ngrok-service} - script de setup interactif ngrok2 et system.d \textit{[Bash]}
  \item \textbf{Micro-Wordpress} - micro-services PHP exposés automatiquement via l'API REST Wordpress \textit{[PHP]}
  \item \textbf{CV} - ce cv est aussi disponible sur mon Github \textit{[LaTeX]}
\end{itemize}

% Reference Test
%\datedsubsection{\textbf{Paper Title\cite{zaharia2012resilient}}}{May. 2015}
%An xxx optimized for xxx\cite{verma2015large}
%\begin{itemize}
%  \item main contribution
%\end{itemize}

\section{\faCogs\ Talents}
\begin{itemize}[parsep=0.5ex]
  \item \textbf{Languages}: Node == Typescript == Javascript == ES6 > PHP == Python > Go
  \item \textbf{Platformes}: Linux == macOs > Windows
  \item \textbf{Développement}: Web \textit{[Typescript/PHP]} == Serveur \textit{[Typescript/Python/PHP/Go]} == OPS \textit{[Docker/AWS]}
\end{itemize}

\section{\faInfo\ Autre}
\begin{itemize}[parsep=0.5ex]
  \item \textbf{Blog}: https://nodend.com
  \item \textbf{GitHub}: https://github.com/jukefr
  \item \textbf{Languages}: English (native) == Français (natif) > Español (poquito)
\end{itemize}

%% Reference
%\newpage
%\bibliographystyle{IEEETran}
%\bibliography{mycite}
\end{document}
